\documentclass[conference]{IEEEtran}
\IEEEoverridecommandlockouts
% The preceding line is only needed to identify funding in the first footnote. If that is unneeded, please comment it out.
\usepackage{cite}
\usepackage{amsmath,amssymb,amsfonts}
\usepackage{algorithmic}
\usepackage{graphicx}
\usepackage{textcomp}
\usepackage{xcolor}
\usepackage{tabularx,booktabs}
\def\BibTeX{{\rm B\kern-.05em{\sc i\kern-.025em b}\kern-.08em
    T\kern-.1667em\lower.7ex\hbox{E}\kern-.125emX}}
\begin{document}

\title{Before Order - A menu information providing chatbot service}

\author{\IEEEauthorblockN{1\textsuperscript{st} Kang Minju}
\IEEEauthorblockA{\textit{dept. Information System} \\
\textit{Hanyang Univ.}\\
Seoul, South Korea \\
kmj1997@gmail.com}
\and
\IEEEauthorblockN{2\textsuperscript{nd} Lim Hyojin}
\IEEEauthorblockA{\textit{dept. Information System} \\
\textit{Hanyang Univ.}\\
Seoul, South Korea \\
hyojin12288@gmail.com}
\and
\IEEEauthorblockN{3\textsuperscript{rd} Choo Yedeun}
\IEEEauthorblockA{\textit{dept. Information System} \\
\textit{Hanyang Univ.}\\
Seoul, South Korea \\
cydddd1221@gmail.com}
\and
\IEEEauthorblockN{4\textsuperscript{th} Christian Gärtner}
\IEEEauthorblockA{\textit{dept. Computer Science} \\
\textit{Hanyang Univ.}\\
Esslingen am Neckar, Germany \\
christian.gaertner97@gmail.com}
}

\maketitle

\begin{abstract}
Globalization attracted many newcomers to South Korea but our restaurant services have not kept up with the needs of foreign customers. Foreigners still have difficulties reading and understanding menus in traditional but also modern Korean restaurants. Therefore, we will develop ‘BeforeOrder’ by using the ‘KakaoTalk’ chatbot API and provide foreigners with a service to conveniently receive information about menus and dishes throughout their journey.
\end{abstract}

\begin{IEEEkeywords}
translator, chatbot
\end{IEEEkeywords}

\section{Introduction}
Many foreigners are often lost and embarrassed when eating out at restaurants as they can neither read the foreign menus nor can they understand what ingredients are used for specific dishes. Those things can make it hard to try and enjoy foreign food especially for people who have allergies and therefore have to ask the staff every time. But staff members, especially in more rural areas, often do not speak or understand English and cannot provide sufficient information. On the other hand, restaurants often simply do not have the space in menus to list the name of dishes and its ingredients in both, the foreign and the English language.

 Whenever foreigners face this situation, they have to search and retrieve all the different information manually. However, foreigners who do not know ‘Hangeul’ will not be able to search directly on Google because they do not understand the dish names in a menu and often also do not have access to a ‘Hangeul’ keyboard. Even if someone searches for the information, it may not be properly categorized and recognized by a translator as they often do not distinguish dishes and it may happen that they simply translate the individual letters into English. Eventually the user may need to search more and more and ends up leaving the restaurant and eat someplace else. 

 We will try to solve this problem with ‘KakaoTalk’, a chat application that is essential in Korea. Since 2013, Kakao has provided an automated response API through KakaoTalk Plus Friends. Plus Friends can be added as a friend in KakaoTalk to receive various contents, benefits and information. Using this method, the user can conveniently install and delete the service provided by us and also use the functionality with an intensively tested and user-friendly interface.

 A chatbot in general describes an artificial intelligence-based program, that analyzes the conversation with a user. Users of the messenger can naturally send messages or ask questions to a chatbot which in return provides a service that responds to the user and seems to communicate with him. There are multiple reasons as to why KakaoTalk chatbots are getting attention. First, it has a familiar UX design and developing a new UX can be very time-consuming and users may have difficulties adapting to a new design. Second, people tend to be reluctant to download new apps for a variety of reasons. Third, using a popular and well-known messenger improves accessibility. We think using a chatbot will increase the popularity of the service and make it more appealing to customers in our target audience. 

 [Before Order] target customers are foreigners that are not familiar with the Korean language and alphabet and therefor having problems eating out. The service is especially aimed at exchange students and employees who plan to stay in Korea in the long term instead of tourists that will mostly eat in international restaurants. [Before Order] follow this process: 
\begin{enumerate}
\item befriend [Before Order] as a Plus Friend in KakaoTalk
\item provide information about the menu to the chat bot in form of images or text
\item receive information about a dish by choosing from the menu list provided by [Before Order]
\end{enumerate}
That way, customers can 
\begin{itemize}
\item reach the service anytime
\item receive information about dishes without searching multiple times
\item do not have to register up for any service except befriending [Before Order] on KakaoTalk
\end{itemize}

\begin{table}
\caption{User Roles}
\begin{tabularx}{\linewidth}{|X|X|X|}
\toprule
Role                & Name              & Task and description \\
\midrule
User                & Choo Yedeun       & Assumes user point of view and guarantee a usable and user-friendly software \\
Customer            & Lim Hyojin        & Provides detailed software requirements and reviews delivered features \\
Software Developer  & Christian Gärtner & Responsible for writing and developing software features and satisfying the needs of the costumer    \\
Development Manager & Kang Minju        & Supervises the development of the service, managing of deadlines and evaluation of software features
\end{tabularx}
\end{table}
\section{Requirements}
\subsection{Functional requirements}The system will be usable on a smartphone – In order to enable the user to use the provided service while the user is moving around the system has to be reachable from a mobile platform. The user also shall be able to conveniently use a camera to take pictures of the menu without having to type and search for one dish name at a time.
The system should be usable on notebooks, tablets and stationary computers – The user shall be able to use the service at home or similar environments to be able to gather information about restaurants and dishes and help with the selection of a place to eat.
The system will run on Android, iOS and Windows 7/8/10 – The system shall run on the most popular operation systems in order to be accessible for as many users in the target audience as possible.
The system will run within a common third-party software – The user shall be provided with an intensively tested and user-friendly interface. In addition, the user should not have to download a separate application to use the functionalities of this system but instead use this service as an extension to a familiar and popular application.
The system shall provide contact information to enable the user to contact the support – The user shall be able to receive help in case of questions or problems with the service. In addition, the user should be able to report any bugs that he finds while using the service or give feedback about it.
The user shall be able to save information about a dish – The user should be able to retrieve information about dishes that he once searched for without having to query the search again. The user should also be able to read the information after closing and reopening the application.
The system shall support the Korean alphabet as input
The system will display information about dishes in the English language and alphabet
The user shall be able to select a dish in the analyzed menu and receive detailed information for this specific dish
The system should provide information about the dishes in form of used ingredients, possible allergies and level of spiciness
The system shall accept pictures and text as input – The user should be able to manually search for a dish if the analyzing of the picture returns no matches
The system should distinguish between dish names and random letter clusters – The user may input pictures without a menu or the pictures that include random text in addition to the menu

\subsection{Working mechanisms}
1. Register service
User will search for [Before Order] in the KakaoTalk Plus Friend list and add the chatbot as a friend. Registration process is finished.
2. Input information
Take a picture of a menu written in Korean alphabet and send it to the [Before Order] chat bot.
3. Recognition process
The [Before Order] service analyzes the input picture or text and tries to recognize the dishes.
4. Select information
If the user sent a whole menu he can select a single dish out of the list that [Before Order] provides and receive more detailed information for that specific dish.
5. Display detailed information
[Before Order] provides the user with detailed information about the requested dish using public APIs.


\end{document}
