\section{Introduction}
Many foreigners are often lost and embarrassed when eating out at restaurants as they can neither read the foreign menus nor can they understand what ingredients are used for specific dishes. Those things can make it hard to try and enjoy foreign food especially for people who have allergies and therefore have to ask the staff every time. But staff members, especially in more rural areas, often do not speak or understand English and cannot provide sufficient information. On the other hand, restaurants often simply do not have the space in menus to list the name of dishes and its ingredients in both, the foreign and the English language.

 Whenever foreigners face this situation, they have to search and retrieve all the different information manually. However, foreigners who do not know ‘Hangeul’ will not be able to search directly on Google because they do not understand the dish names in a menu and often also do not have access to a ‘Hangeul’ keyboard. Even if someone searches for the information, it may not be properly categorized and recognized by a translator as they often do not distinguish dishes and it may happen that they simply translate the individual letters into English. Eventually the user may need to search more and more and ends up leaving the restaurant and eat someplace else. 

 We will try to solve this problem with ‘Facebook Messenger’, a chat application that is supported by Facebook. Since 2016, Facebook has provided an Chatbot API through using Facebook Messenger. User can connect to the chatbot through the chatbot's Facebook page to receive various contents, benefits and information. Developers can make own Facebook Page for creating their chatbot, and user can access to the chatbot only by clicking function 'send message' in profile of the page. Using this method, the user can conveniently get the service provided by us and also use the functionality with an intensively tested and user-friendly interface.

 A chatbot in general describes an artificial intelligence-based program, that analyzes the conversation with a user. Users of the messenger can naturally send messages or ask questions to a chatbot which in return provides a service that responds to the user and seems to communicate with him. There are multiple reasons as to why Facebook chatbots are getting attention. First, it has a familiar UX design and developing a new UX can be very time-consuming and users may have difficulties adapting to a new design. Second, people tend to be reluctant to download new apps for a variety of reasons. Third, using a popular and well-known messenger improves accessibility. We think using a chatbot will increase the popularity of the service and make it more appealing to customers in our target audience. 

 \emph{Before Order} target customers are foreigners that are not familiar with the Korean language and alphabet and therefor having problems eating out. The service is especially aimed at exchange students and employees who plan to stay in Korea in the long term instead of tourists that will mostly eat in international restaurants. \emph{Before Order} follow this process: 
 
\begin{enumerate}[label=Step \arabic*:, leftmargin=1.5cm]
\item follow or enter the \emph{Before Order} Facebook Page and click 'send Facebook Message' to send message 
\item provide information about the menu to the chat bot in form of images or text
\item receive information about a dish by choosing from the menu list provided by \emph{Before Order}
\end{enumerate}

That way, customers can 
\begin{itemize}
\item reach the service anytime
\item receive information about dishes without searching multiple times
\item do not have to register up for any service 
\end{itemize}

\begin{table}[htb]
\caption{User Roles}
\begin{tabularx}{\linewidth}{|X|X|X|}
\toprule
Role                & Name              & Task and description \\
\midrule
User                & Choo Yedeun       & Assumes user point of view and guarantee a usable and user-friendly software \\
Customer            & Lim Hyojin        & Provides detailed software requirements and reviews delivered features \\
Software Developer  & Christian Gärtner & Responsible for writing and developing software features and satisfying the needs of the costumer    \\
Development Manager & Kang Minju        & Supervises the development of the service, managing of deadlines and evaluation of software features
\end{tabularx}
\end{table}
\FloatBarrier