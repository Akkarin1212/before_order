\section{Development Environment}
\subsection{Choice of software development platform}
\begin{table}[htb]
\caption{Development Tools}
\begin{tabularx}{\linewidth}{|X|X|}
\toprule
Tool & Usage \\
\midrule
JAVA & We’ve chosen Java as a overall basic language in developing \emph{Before Order} because it is one of the general, object-oriented language and It is a language our team members are all familiar with.\\
Github	& We decided to use Github because it provides the necessary management functions for software development including the basic functions of the Git such as bug tracking, functional requests, task management and etc. Also, we can easily share and develop program sources together with our team members.\\
Visual Studio & Visual studio is an integrated development environment(IDE) program made from Microsoft that includes many features as compiler, editor and debugging. We are going to utilize many various necessary functions of visual studio in practical implementation.\\
Slack & Slack is a collaborative tool that enables more efficient work and communication while developing software. We are going to take advantage of the work messenger functionality of slack and its convenient drag-and-drop format file sharing
\end{tabularx}
\end{table}
\FloatBarrier

\subsection{Software in use}
The similar software can be referred to as Samsung \emph{Bixby vision}. \emph{Bixby vision} automatically extracts text messages through camera lens and translate them into user-set languages in real time. In order to use \emph{Bixby vision} menu analysis, we can just bring the camera lens to the menu. However there are two major differences that suggest why our \emph{Before Order} is better than ‘Bixby vison’. First, \emph{Bixby vision} only provide literal interpretation of food name. Therefore people are more likely to have difficulties not knowing what exactly it means. However \emph{Before Order} give more clear information as it aims to provide photo, detailed explanation and name of the food/menu in english. Lastly \emph{Bixby vision} is limited to only samsung product but our \emph{Before Order} is available on all platforms and devices since it will be implemented in chat-bot.

\subsection{Cost estimation}
\begin{table}[htb]
\caption{Development Tools}
\begin{tabularx}{\linewidth}{|X|X|}
\toprule
Tool & Cost \\
\midrule
Azure Bot Service & Free \\
SQL database \& server & 16870.76 Won  for 5GB / 10 DTU\\
Computer Vision API & 2811.63 Won for 1000 transactions\\
\end{tabularx}
\end{table}
\FloatBarrier

The OS version in which we will develop the program is Window 10, and We will use our own laptop for developing our service.

We will also use cloud computing service using Azure cloud platform. 
The service we are using for making our service is:
\begin{enumerate}
\item SQL Database 
\item Azure Bot Service
\item Computer Vision API : OCR api
\end{enumerate}
We will connect our Azure Bot Service with Facebook messenger platform, and make senario in the Azure Bot Service. 
In order to store and manage the menu data, we will connect our Bot service with the SQL Database from Azure. We will make database for storing korean and english name of the menu and the details about menu (picture, ingredients, taste, spicyness, etc).
Computer vision api will provide us texts from the picture that users send through the bot.
